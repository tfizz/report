\documentclass[conference]{IEEEtran}
\usepackage{cite}
\begin{document}

\title{Phishing Detection Using Email Honeypot\\}

\author{\IEEEauthorblockN{Nick Radford}
\IEEEauthorblockA{\textit{School of Computing} \\
\textit{Queen's University}\\
Ontario, Canada \\
nick.rashford@queensu.ca}
\and
\IEEEauthorblockN{Babatunde Bamgbose}
\IEEEauthorblockA{\textit{Electrical and Computer Engineering} \\
\textit{Queen's University}\\
Ontario, Canada \\
bamgbose.babatunde@queensu.ca}
}

\maketitle

\begin{abstract}
    Phishing has become an increasing problem in today’s society. They target and exploit the vulnerabilities that are present in systems through the end users. Email is one of the most common method through which phishing attacks are carried out and this is due to the large number of businesses and people using this mode of communication, the ease of access to email addresses of the owners over the internet with or without their consent and the negligible cost and effort it takes in sending the emails. This leaves users vulnerable to this form of attack in which their personal information can be stolen. In this paper, we propose a method for detecting phishing attacks associated with email by using an email honeypot system. This system is intended to automatically receive, categorize, analyze and archive unsolicited mails.
\end{abstract}

\begin{IEEEkeywords}
    phishing, email, spamming, honeypot
\end{IEEEkeywords}

\section{Introduction}
Various technical means of protecting IT infrastructure are available, however, the end users remain the weakest point in computer security. Many of the security attacks named social engineering are based on this assumption. Phishing is one of the popular practices used among computer criminals \cite{phigaro} and it aims at exploiting the weakness that are found in processes of systems through the system users \cite{khonji}. An example is a system that has a sufficiently secure password protection mechanism and users leaking their passwords through a Hypertext Transfer Protocol link provided to them by an attacker asking them to update their password, which threatens the security of the system.

\bibliographystyle{IEEEtran}
\bibliography{references}
    
\end{document}
